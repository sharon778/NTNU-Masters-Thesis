%===================================== CHAP 5 =================================

\chapter{Conclusions and future work}


% ============================ section ===============================
\section{Main findings}



% ============================ section ===============================
\section{Discussion}



% ============================ section ===============================
\section{Future directions}

We wish to symbolically define what programatically constitutes the sub-symbolic principles for the emergence of the neural mechanisms studied in this thesis. Furthermore, we wish to combine these mechanisms, investigating possible implications in the dual-network memory architecture. This could suggest further directions for the field of ANNs.

\section{Speculations about high-level cognition}
Continuous neural activity, perhaps even entirely deterministic, in the mammalian brain may actually be constituted entirely by the sum of accumulated experience. However, there may be some crucial parts responsible for mediating a type of focus in directing activity towards certain associations. Whether such an explicit mechanism exists, and whether it too may be entirely deterministic, remains a philosophical question. It should be noted that it lies in our command to recall memories, and even to imagine.

\section{Notes}

The segmentation issue due to lack of plasticity as previously mentioned is most likely still present in today's deep neural networks. -> Should be addressed.



We believe that a central aspect in continuing to advance the frontier of deep learning is to investigate how high-level level cognitive behaviour and functionality may emerge in ANNs. More specifically, we wish to further investigate the mechanism of reasoning over different memories, potentially providing insights for attaining greater plasticity and generalisation in ANN models. A crucial aspect of being able to combine different memories is simply remembering what has previously been learned. Therefore, the foremost goal of the thesis is to investigate a dual-network memory architecture, such as the model which \cite{Hattori2014} proposes. Furthermore, we wish to study different variations of such a model. This includes experiments where other successful and novel approaches within the field are tested, using the dual-network memory architecture as a framework for studying potential emergent neural mechanisms and network behaviour.
One interesting aspect in such experiments is how recurrency supports specific functionality and emergence within a network. In other words: How recurrence is coupled to both avoiding catastrophic interference and forgetting, and how recurrence adds dimensions to the information processing capabilities of some ANNs. Furthermore, we wish to investigate alternative implementations of the neocortical network and the associated implications for long-term memory.

\cleardoublepage