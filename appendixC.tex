\section*{\begin{center}{Appendix C}\end{center}}
\addcontentsline{toc}{section}{Appendix C}
%$\\[0.5cm]$

\section*{Architectural Overview}

% [git repo]
The entire code base of the implementation is contained in the public git-repository https://github.com/williampeer/DeepBytes. Furthermore, this contains a short readme for how to setup a system such that the model and experiments may be run.

% might want to refactor this one
\section*{Hippocampal code example}

\begin{Verbatim}[fontsize=\small]
import theano
import theano.tensor as T
import numpy as np
import Tools

class HPC:
    # the parameters being omitted for simplicity
    def __init__(self, ..parameters):
        # variables are bound to the object instance (...)
    
        # ============== SETUP ==================
        input_values = np.zeros((1, dims[0]), dtype=np.float32)
        self.input_values = theano.shared(name='input_values', 
            value=input_values.astype(theano.config.floatX), 
            borrow=True)
                                          
        # the remaining activation values are bound (...), and 
        #   weight matrices are setup similarly:
        input_ec_weights = Tools.binomial_f(dims[0], dims[1], 
            self.connection_rate_input_ec)
        self.in_ec_weights = theano.shared(name='in_ec_weights', 
            value=input_ec_weights.astype(theano.config.floatX), 
            borrow=True)
                                           
        # (...)
        
        # Theano functions are symbolically defined and bound to 
        #   the object instance, such as:
        local_in_vals = T.fmatrix()
        local_in_ec_Ws = T.fmatrix()
        next_activation_values_ec = T.tanh(local_in_vals.dot(\
            local_in_ec_Ws) / self._epsilon)
        self.propagate_input_to_ec = theano.function(\
            [local_in_vals, local_in_ec_Ws], outputs=None,
            updates=[(self.ec_values, next_activation_values_ec)])

        # wire after kWTA for this layer
        u_prev_reshaped_transposed = T.fmatrix(\
            'u_prev_reshaped_transposed')
        u_next_reshaped = T.fmatrix('u_next_reshaped')
        Ws_prev_next = T.fmatrix('Ws_prev_next')
        # Element-wise operations. w_13_next = w_13 + 
            nu u_3(u_1-u_3 w_13).
        next_Ws = Ws_prev_next + self._nu * u_next_reshaped * \
            (u_prev_reshaped_transposed.T - u_next_reshaped * \
            Ws_prev_next)
        self.wire_ec_dg = theano.function([u_prev_reshaped_transposed,
            u_next_reshaped, Ws_prev_next], 
            updates=[(self.ec_dg_weights, next_Ws)])
            
        # wrapper method for learning:
        def learn(self):
            self.propagate_input_to_ec(self.in_values.get_value(\
                return_internal_type=True), self.in_ec_weights.
                    get_value(return_internal_type=True))
                    
            self.set_ec_values(kWTA(self.ec_values.get_value(...)))
            
            # -> repeat for all layers, using a different function as
            #       outlined for the CA3-layer in Chapter 3.

                                          
\end{Verbatim}

Note that for retrieval of shared variable values in the above code example, I return the internal value - this is to improve efficiency solely, and should not be prone to aliasing with the current implementation.

\subsection*{k-WTA}

k-Winners-Takes-All is implemented very much like the following pseudocode:

\begin{Verbatim}[fontsize=\small]
def kWTA(activation_values):
    sorted_values = activation_values.sort()  # ascending
    threshold = (sorted_values[k-2] + sorted_values[k-1]) / 2
    return filter(activation_values, threshold)

\end{Verbatim}

One addition being the edge case of when all activation values are the same value. Even though this should not occur - the algorithmic edge case is handles by setting k nodes to 1 at random, the remaining to 0. If only a subset of the nodes have the same value, yet having the number of nodes that are above the threshold exceed k; nodes are drawn at random from this subset and set to 0 until the number of nodes that are 1 correspond exactly to k. I.e. the non-optimized pseudo-code may be written the following way,

\begin{Verbatim}[fontsize=\small]
    if(numpy.sum(kWTA_arr) > k):
        while numpy.sum(kWTA_arr) > k:
            kWTA_arr = remove_random_of_smallest(kWTA_arr)

\end{Verbatim}

learn-wrapper, the main method; excerpt [appendix]

\clearpage