\clearpage
\pagenumbering{roman} 				
\setcounter{page}{1}

\pagestyle{fancy}
\fancyhf{}
\renewcommand{\chaptermark}[1]{\markboth{\chaptername\ \thechapter.\ #1}{}}
\renewcommand{\sectionmark}[1]{\markright{\thesection\ #1}}
\renewcommand{\headrulewidth}{0.1ex}
\renewcommand{\footrulewidth}{0.1ex}
\fancyfoot[LE,RO]{\thepage}
\fancypagestyle{plain}{\fancyhf{}\fancyfoot[LE,RO]{\thepage}\renewcommand{\headrulewidth}{0ex}}

\begin{centering}
\subsubsection{TDT4900 - Computer Science, Master's Thesis}

\end{centering}

$\\[0.2cm]$

\begin{centering}
\subsection*{Dual-network Memory Architectures in Artificial Neural Networks}
\end{centering}
\addcontentsline{toc}{chapter}{Abstract}
$\\[0.1cm]$
\begin{centering}
Fall 2015

\end{centering}
$\\[0.025cm]$

\begin{centering}
William Peer Berg
(williapb@stud.ntnu.no)
\\
Supervisor: Keith L. Downing

\end{centering}
$\\[0.025cm]$

\begin{centering}
The Norwegian University of Science and Technology, NTNU
\\
The Department of Computer and Information Science, IDI

\end{centering}
$\\[0.2cm]$

Connectionism and recent advances.
Computational neuroscience as origin?
Dual-network memory architecture; tapping into both.
Catastrophic forgetting in neural networks.
(The symbol grounding problem)
Building upon the work of... \cite{Hattori2014}: novel model, blablabla.
Further towards connectionism with GRUs XOR further towards comp. neuro. w/ some other characteristic.

Introduce deep learning
Quickly introduce the dual-network memory model. establish connection to computer science. 
create a research space for the thesis to fill - mention aspects that haven’t yet been solved, addressed, or argue that they should be further illuminated. CARS. 
briefly mention what’s been done and results.

\clearpage